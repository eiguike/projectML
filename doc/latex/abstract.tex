\begin{abstract}
As empresas cada vez mais se utilizam de diferentes formas para agregar novos clientes à seus produtos. Uma das formas de atingir seus clientes é por campanhas de telemarketing, em que a prospecção de vendas se classficam como: \emph{inbound} (quando o cliente tem a pró-atividade em comprar o produto), \emph{outbound} (quando um intermediário oferece tal produto diretamente pro cliente). É extremamente importante identificar os possíveis clientes para que seja possível maximizar a prospecção de vendas do produto, dessa forma, utilizando de Sistemas de Suporte à Decisão em conjunto com algoritmos de Aprendizado de Máquina, é possível identificar com eficiência quem são os melhores clientes para se oferecer tal produto. Diante deste cenário, esse trabalho apresenta uma análise de técnicas de aprendizado de máquina aplicados na classificação dos melhores clientes. Utilizando uma base de dados real criada por um banco português, em sua campanha de marketing, foi possível realizar experimentos utilizando os algoritmos k-vizinhos mais próximos, regressão logística, redes neurais e máquina de vetores de suporte; resultados obtidos nos nossos experimentos indicaram que o uso de regressão logística é apropriado para construção de um modelo promissor na identificação dos usuários que tem uma maior chance de aderir à venda proporcionada por uma campanha de telemarketing.

\end{abstract}

\begin{IEEEkeywords}
telemarketing bancário; vendas; classificação.
\end{IEEEkeywords}