\section{Introdução}
As empresas cada vez mais se utilizam de diferentes formas para agregar novos clientes à seus produtos. Uma dessas formas é por meio de campanhas de telemarketing, forma que nasceu e cresceu na década 19 com a popularização do telefone. Com as empresas, cada vez mais com uma demanda maior para vender seus produtos, diversos empregos como Operador de Telemarketing foram criados, além de ser uma forma barata de campanha, é uma das formas em que mais clientes são atingidos.

Há duas classificações das prospecções de vendas de telemarketing, sendo elas,  \emph{inbound}, quando é o cliente que realiza a ligação para a empresa com interesse de comprar o produto, e \emph{outbound}, quando o Operador de Telemarketing realiza a oferta de um produto para um cliente. Para ter êxito nas vendas do produto, é imprescindível conhecer o público alvo do produto e também o perfil do cliente que irá ser oferecido, dessa forma, escolher o melhor conjunto de clientes que poderão comprar o produto pode ser considerado parte do conjunto de problemas NP-Difícil \cite{np_dificil}.

Nesta situação, umas das soluções é a utilização de Sistemas de Suporte à Decisão, em que, são sistemas que utilizam um modelo genérico de tomada de decisão e tem como base sua análise em um grande conjunto de variáveis, dessa forma, através da análise dos dados é possível determinar um posicionamento e/ou escolha \cite{administracao_sistemas}.

Muitos algoritmos de classificação são utilizados em Sistemas de Suporte à Decisão, sendo eles o k-vizinhos próximos, regressão logística, árvores de decisão, redes neurais, e ainda os mais recentes, Deep Learning (redes neurais) e máquinas de vetores de suporte que é o estado da arte na área de Aprendizado de Máquina. \cite{rule_extraction}

Neste artigo iremos avaliar o desempenho dos métodos de aprendizado de máquinas: k-vizinhos próximos, regressão logística, redes neurais e máquinas de vetores de suporte. Esses métodos poderão ser utilizados para aumentar a prospecção de venda de produtos, a partir da classificação do perfil dos clientes de uma campanha de telemarketing. O objetivo principal é realizar uma análise dos métodos de classificação em conjunto com os possíveis parâmetros promissores deste cenário.

Este artigo está estruturado da seguinte forma, na Seção II está apresentado a modelagem da base de dados, além de sua origem e características. Na Seção III é apresentado os métodos de aprendizado de máquina, seus parâmetros de configuração e suas características. Na Seção IV encontra-se os resultados obtidos a partir da execução desses métodos de classificação, e por último, na Seção V, são apresentadas as principais conclusões e análises deste trabalho.
