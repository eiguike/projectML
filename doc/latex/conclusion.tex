\section{Conclusões}
Este trabalho apresentou os resultados obtidos de quatro métodos de classificação aplicados ao \emph{Bank Marketing Data Set}\cite{bank_dataset} após várias análises e aplicações de técnicas de Aprendizado de Máquina. 

O classificador \emph{kNN}, além de computacionalmente custoso não obteve resultados satisfatórios, obtendo uma acurácia total de $~51\%$, já os classificadores de Redes Neurais e SVM obtiveram bons resultados, acurácia total em ambos $~87\%$, aliados à um bom desempenho computacional, sendo custoso apenas a otimização dos parâmetros, exigindo uma busca em \emph{grid} extensiva. Por fim, o melhor classificador foi o de Regressão Logística, com uma acurácia de $97.54\%$. Por ser uma técnica de baixa complexidade computacional é uma excelente escolha para este conjunto de dados dada a sua alta precisão de classificação.

Trabalhos futuros podem incluir análises focadas em identificar os principais fatores que levam um cliente aderir ou não à campanha de telemarketing, assim fazendo com que os bancos saibam os pontos exatos onde devem investir, visando aumentar a participação de clientes, e, consequentemente, seus lucros.

A realização desse trabalho envolveu a aplicação de inúmeros conceitos aprendidos na disciplina de Aprendizado de Máquina e forçou os autores a fortificar seus conhecimentos e verificar, na prática, as várias variáveis incluídas no processo de desenvolver um modelo preditivo de dados.